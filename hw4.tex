\documentclass[11pt]{article}
\usepackage{fullpage, amsmath, hyperref, epsfig,supertabular}
\usepackage[titletoc]{appendix}
\usepackage{tikz}
\usepackage{qtree}

\begin{document}
\begin{center}
{\bf\Large CMPS 142 --- Spring 2017 --  Homework 4}
\end{center}
Alexander Morsey(ammorsey), Jared Jensen(), Sinclert Perez()

\section*{1.}
After plotting the points into a 2 dimensional spaces (there are only 2 features)
we can clearly see that the hyperplanes containing the support vectors of each class
are defined the following equations:

Class "+1": \begin{equation}
            x_1 + x_2 - 2 = 0
            \end{equation}

Class "-1": \begin{equation}
            x_1 + x_2 - 1 = 0
            \end{equation}


The separation plane is the one between those two. Its equation is:

\begin{equation}
x_1 + x_2 - 1.5 = 0
\end{equation}


The geometric margin is computed applying the following formula:

\begin{equation}
min_i \frac {y_i(w^Tx_i+b)}{||w||}
\end{equation}

In our case, plotting the different instances into the formular we got that the minimum is:

\begin{equation}
\frac {1 * (0.5)}{\sqrt{2}} = 3.53
\end{equation}
\section*{2.}
Cross validation using several different kernels.
\subsection*{RBF}
\begin{flushleft}
\begin{supertabular}{|l|l|l|l|}
	\hline
	Test & $\gamma$ & C & Result(\%)\\
	\hline
	1. & 0.01 & 0.1 & 60.596\\
	\hline
	2. & 0.1 & 0.1 & 73.136\\
	\hline
	3. & 1 & 0.1 & 87.503\\
	\hline
	4. & 10 & 0.1 & 89.372\\
	\hline
	5. & 0.01 & 1 & 73.462\\
	\hline
	6. & 0.1 & 1 & 87.22\\
	\hline
	7. & 1 & 1 & 92.371\\
	\hline
	8. & 10 & 1 & 93.067\\
	\hline
	9. & 0.01 & 10 & 87.112\\
	\hline
	10. & 0.1 & 10 & 92.154\\
	\hline
	11. & 1 & 10 & 93.675\\
	\hline
	12. & 10 & 10 & 93.219\\
	\hline
	13. & 0.01 & 100 & 91.437\\
	\hline
	14. & 0.1 & 100 & 93.349\\
	\hline
	15. & 1 & 100 & 93.893\\
	\hline
	16. & 10 & 100 & 92.567\\
	\hline

\end{supertabular}
\end{flushleft}

\subsection*{Polynomial}
\begin{flushleft}
\begin{supertabular}{|l|l|l|l|l|}
	\hline
	Test & $\gamma$ & C & coef & Result(\%)\\
	\hline
	1. & 1 & 100 & 0 & 54.206\\
	\hline
	2. & 1 & 100 & 1 & 64.269\\
	\hline
	3. & 1 & 100 & 3 & 55.661\\
	\hline
	4. & 1 & 100 & -1 & 72.679\\
	\hline
	5. & 2 & 100 & -1 & 67.494\\
	\hline
\end{supertabular}
\end{flushleft}
Running the tests with the polynomial kernel was incredibly slow, so we could not fully test the parameters.

\subsection*{Linear}
\begin{flushleft}
\begin{supertabular}{|l|l|l|}
	\hline
	Test & Cost & Result(\%)\\
	\hline
	1. & 1 & 88.937\\
	\hline
	2. & 10 & 80.221\\
	\hline
	3. & 100 & 72.702\\
	\hline
\end{supertabular}
\end{flushleft}
Of what we tested RBF did the best. Hypothetically the polynomial kernel should do better, but we couldn't get it to do better.
\section*{3.}
	We need a dataset with 3 criteria:
	\begin{enumerate}
		\item There is a small decision tree that correctly classifies the data
		\item The greedy decision tree construction algorithm constructs a larger tree
		\item There are no ties in the construction process.
	\end{enumerate}

	Here is the training dataset that we came up with.

	\begin{tabular}{c|c||c}
		$X_1$&$X_2$&$Y$\\
		1&1&1\\
		0&1&0\\
		1&1&1\\
		1&0&1\\
		0&0&0\\
		1&0&1\\
	\end{tabular}

	Simple tree that correctly classifies the data:

	\Tree[[.{$X_1$ = 1} [.{Predict 1} ]]
          [.{$X_1$ = 0} [.{Predict 0} ]]]

    Greedy construction:

    Step 1: Initial feature with the highest error rate is $X_2$. Split based off of that.

    \Tree[[.{$X_2$ = 1} ]
   	      [.{$X_2$ = 0} ]]

   	On the left side, the data is

   	\begin{tabular}{c|c||c}
		$X_1$&$X_2$&$Y$\\
		1&1&1\\
		0&1&0\\
		1&1&1\\
	\end{tabular}

	And on the right side, the data is

	\begin{tabular}{c|c||c}
		$X_1$&$X_2$&$Y$\\
		1&0&1\\
		0&0&0\\
		1&0&1\\
	\end{tabular}

	For both sides, X1 now has a higher error rate. Split off that.

	\Tree[[.{$X_2$ = 1}
	      	[.{$X_1 = 1$} [.{Predict 1} ]]
	      	[.{$X_1 = 0$} [.{Predict 0} ]]]
   	      [.{$X_2$ = 0}
   	        [.{$X_1 = 1$} [.{Predict 1} ]]
	      	[.{$X_1 = 0$} [.{Predict 0} ]]]]
\end{document}
